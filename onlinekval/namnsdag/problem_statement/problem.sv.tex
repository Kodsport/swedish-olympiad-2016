\problemname{Namnsdag}

Din kompis gillar pengar. När din kompis har namnsdag så får hen pengar av sin snälla moster,
din kompis tycker därför att namnsdagar är jätteskoj.

Det råkar vara din kompis namnsdag idag, och din kompis funderar på hur hen skulle kunna
undvika att behöva vänta ett helt år innan det är dags för namnsdag igen. Du har blivit tillfrågad
om att hjälpa till.

Din kompis planerar att byta ut exakt en bokstav i sitt namn för att ha namnsdag så
snart igen som möjligt. Givet en lista på namnsdagar för dagarna det kommande
året, avgör hur snart hen kan ha namnsdag igen. Det är förbjudet att ta bort eller
lägga till bokstäver, du får bara byta ut exakt en bokstav mot en annan.

Om det inte går att hitta ett namn att byta till som gör att det är dags för
namnsdag tidigare så får din kompis nöja sig med sitt namn och därmed vänta ett
helt år till.

\section*{Input}
Indata börjar med en sträng på en rad, namnet på din kompis.
Sedan följer ett heltal $N$ på en rad, antalet dagar det kommande året.
Efter det följer namnsdagarna för de $N$ kommande dagarna, ett namn per rad.

Den $N$:e namnet kommer alltid vara namnet på din kompis,
eftersom hen har namnsdag idag och därmed också om ett år igen. Namnen i indata
består endast av tecken \texttt{a-z}, innehåller inga mellanslag och är max 10 tecken långa. Alla namn i indata är olika.


\section*{Output}
Ditt program ska skriva ut ett enda tal: antalet dagar tills det är dags för
namnsdag igen, om du hjälper din kompis att fuska med sitt namn. 

\section*{Förklaring av exempel}
I det första indataexemplet så heter din kompis \texttt{anna}. Hon kan skapa namnet
\texttt{anja} genom att byta ut det andra \texttt{n}et i sitt namn mot ett \texttt{j},
 och således ha namnsdag två dagar tidigare än väntat. Svaret är alltså 3.
I det andra indataexemplet så går det inte att byta ut exakt en bokstav mot en annan
så att din kompis \texttt{jan} har namnsdag tidigare, han får alltså nöja sig med att
vänta ett helt år, det vill säga 3 dagar.

\section*{Poängsättning}
Din lösning kommer att testas på en mängd testfallsgrupper. För att få poäng för en grupp
så måste du klara alla testfall i gruppen.

\begin{tabular}{| l | l | l | l |}
\hline
Grupp & Poängvärde & Gränser & Övrigt \\ \hline
1     & 53         &  $1 \le N \le 50$ & \\ \hline
2     & 47         &  $1 \le N \le 10\,000$ & \\ \hline
\end{tabular}
