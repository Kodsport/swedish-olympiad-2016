\problemname{Robotoptimering}

There's a robot in an $N \times M$ grid, where some unit squares are \textbf{blocked} so that the robot cannot walk on the square. Now she wants to move to another square, and have asked her owner to program her to do this.
This owner happens to be you.

To transfer the code to a robot from your computer requires a lot of energy, so you want to make the program as short as possible - this means using as few commands as possible.
As everyone knows, the robot programming language looks as follows:

\begin{description}
  \item[\texttt{forward}] Move the robot forward a single step in its current direction.
  \item[\texttt{right}] Rotate $90$ degrees clockwise.
  \item[\texttt{left}] Rotate $90$ degrees counterclockwise.
  \item[\texttt{for X \{ A1 A2 ... An \}}] Repeat the commands \texttt{A1}, \texttt{A2}, ..., \texttt{An} $X$ times.
  \item[\texttt{call X}] Add your current position in the program to the top of the call stack, then jump to the instruction with label \texttt{X}.
  \item[\texttt{return}] Jump to the last added position in the call stack, and remove it.
  \item[\texttt{gotoblocked X}] Jump to the instruction with label \texttt{X}, if the square in front of the robot is blocked.
\end{description}

A \emph{label} has the syntax \texttt{labelname:}, and may not contain any whitespaces. A label cannot be inside a loop. The execution starts at the label called \texttt{main}.

\section*{Example program}

\begin{verbatim}

walkandreturn:
  for 100 {
    forward
  }
  gotoblocked done
  right
  right
  for 100 {
    forward
  }
done:
  return

main:
  for 100 {
    call walkandreturn
    right
  }
\end{verbatim}

Note that there is no concept of a function; if the return was not present in \texttt{walkandreturn}, the execution would continue down, running \texttt{main}.

When the robot tries to move towards the edge of the grid, or a blocked square, it doesn't move. When the robot reaches her target square, she wins, even if she doesn't stop on the square.

\section*{Input}
Input consists of 10 different grids, which you can download from attachments at the bottom of the page. In the results page,
the testcases are ordered lexicographically. This means that group 1 is \texttt{robot\_concentric.in}, group 2 is \texttt{robot\_cross.in}, etc.

Each grid is formatted as follows:

The first line contains the testcase name.

The next line contains two integers $1 \le R \le 1000$ and $1 \le C \le 1000$, the number of rows and columns in the grid.

Each square is one of:
\begin{description}
  \item[\texttt{.}] free square
  \item[\texttt{\#}] blocked square
  \item[\texttt{M}] goal square
  \item[\texttt{<>\textasciicircum{}v}] start square, indicating the direction of the robot (in the order left, right, up, down).
\end{description}

\section*{Output}
For the given maze, output a valid program on the form described above, that moves the robot from its starting position to the end position.

\section*{Additional details}
The judge runs $10^8$ steps before it deems that your program won't find the goal.
Every step corresponds to one non-empty line in the program, including labels. As soon as the robot
moves on top of the goal square, the execution is stopped, and you receive an \texttt{Accepted} verdict
for said testcase.

Every line must either be empty or contain exactly one token, which are described at the top of the page. \\
Notably, loops \textit{must} be formatted in the following way:
\begin{verbatim}
for X {
  ...
}
\end{verbatim}
Additionally, $X$ must satisfy $1 \leq X \leq 10^8$.\\
This means that the following loops are \textit{not} valid:
\begin{verbatim}
for X
{
  ...
}
\end{verbatim}
\begin{verbatim}
for X { ... }
\end{verbatim}


If a jump is initiated because of a \texttt{gotoblocked} instruction, the state of every loop is reset. \\
If a jump is initiated because of a \texttt{call} instruction, the state of every loop is saved on the call stack, and they are then reset.
If a \texttt{return} statement is executed, the state of all loops will be set to the saved state on the top of the call stack.


The judges guarantee that there exists a program which attain full score under these constraints.

\section*{Submission}
To make it easier to submit your solution, a sample submission is provided, which you can download from attachments at the bottom of the page.
This submission will get 0 points, but contains valid programs.

\section*{Scoring}
The scoring is based on the length of your program. The length is the number of times you use one
of the listed commands in the your program. Our example program has length 11. In particular, labels
do not contribute to the length.

If your length on a particular test case is $L$ and the shortest length on this case found by the
judges is $B$, then your score on the testcase is

\[ 10 (1 - (\frac{L - B}{L})^2)\]
