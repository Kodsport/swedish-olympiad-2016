\newcommand\version{v1}
\problemname{Videoklipp}
På en populär hemsida så kan de $N$ deltagarna i KATT titta på videoklipp mellan deras problemlösning.

På siten finns det $K$ roliga videoklipp med katter som hoppar runt på ett tangentbord, numrerade mellan $0$ och $K - 1$.
När man tittat på en av videoklippen följer ett förslag på nästa kattvideo, som du såklart klickar på och börjar titta på.

För varje tävlanden får du den ursprungliga kattvideon hen tittade på. Avgör vilken den $M$:te kattvideon varje tävlanden tittar på blir.

\section*{Exempel}
Totalt har vi $N = 2$ tävlanden, och $K = 4$ kattivdeos.
Förslagen för varje video är $(3, 2, 1, 0)$. Den första tävlanden börjar med att titta på video 3, och den andra börjar på video 1.
Vi undrar vilka de $M = 3$:e videoklippen blir för de två tävlanden.

Den första tävlanden börjar med att titta på video 3. Därefter tittar hon på video 0, och sist återigen på video 3.

Den andra börjar på andra videon, som länkar till den tredje. Video 1 har video 2 som förslag, så han hamnar till slut på video 1 återigen.

\section*{Uppgift}
Din uppgift är att beräkna, för varje tävlanden, vad den $M$:te videon hen tittar på blir.
Du ska implementera funktionerna \texttt{videos(K, M, S)} och \texttt{clip(I)}.

\begin{itemize}
  \item \texttt{videos(K, M, S)} - this function will be called exactly once by the judge at the start of the execution.
  \begin{itemize}
    \item \texttt{K}: antalet roliga kattvideos.
    \item \texttt{M}: antalet videos varje tävlanden ska kolla på.
    \item \texttt{S}: en array med $K$ element. \texttt{S[i]} ($0 \le i < N$) innehåller förslaget för video $i$.
		\item $0 \le S[i] < K$
    \item Funktion har inget returvärde
  \end{itemize}
\end{itemize}

\begin{itemize}
  \item \texttt{clip(I)} - denna funktion anropas en gång för varje deltagare.
  \begin{itemize}
    \item \texttt{I}: ett tal mellan $0$ och $K - 1$; videon som tävlanden börjar titta på.
		\item Funktionen ska returnera den $M$:te videon som deltagaren kommer titta på, om hon börjar på video $I$.
  \end{itemize}
\end{itemize}

\section*{Delpoäng}
Problemet består av flera grupper av testfall. Varje grupp ger ett visst antal poäng och för att klara det måste du klara alla testfall i gruppen.

Låt $N$ vara antalet anrop till \texttt{clip(I)}.
\begin{tabular}{|l|l|l|}
  \hline
  \textbf{Grupp} & \textbf{Poäng} & \textbf{Gränser} \\ \hline
  1 & 15 & $N, K \le 100\,000$, $2 \le M \le 10^9$, den $i$:te videon kommer enbart föreslå en video $j$ så att $j \le i$. \\ \hline
  2 & 15 & $N, K \le 100\,000$, $2 \le M \le 10^9$, den $i$:te videon kommer föreslå video $i + 1$ (video $K - 1$ föreslår video 0). \\ \hline
  3 & 15 & $N, K \le 1\,000$ $2 \le M \le 1\,000$ \\ \hline
  4 & 15 & $N, K \le 1\,000$, $2 \le M \le 10^9$ \\ \hline
  5 & 40 & $N, K \le 100\,000$, $2 \le M \le 10^9$ \\ \hline
\end{tabular}

\section*{Indataformat}
Exempeldomaren läser indata i följande format:

\begin{itemize}
  \item rad $1$: \texttt{K M}
  \item rad $2$: \texttt{S[0] ... S[K - 1]}
  \item rad $3$: \texttt{N}: antalet anrop till \texttt{clip(I)}.
  \item rad $4$ \texttt{I1 ... IN}: parametrarna till de $N$ anropen till \texttt{clip(I)}.
\end{itemize}

\section*{Utdataformat}
Exempeldomaren skriver ut $N$ rader med returvärdena från \texttt{clip(I)}.
