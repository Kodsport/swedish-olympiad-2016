\newcommand\version{v1}
\problemname{T-shirts}
During the Programming Olympiad finals, each of the $N$ contestants is always given a t-shirt as is customary. However, the judges
are sometimes a bit stressed out with making last minute changes to the problem sets (only changes, mind you -- the judges
never wait until the day before the contest to make a problem).

This means that when the judges order the t-shirts, they may not look \emph{that} carefully on what t-shirt
sizes the contestants have. After all, who can distinguish
a size \texttt{XS} t-shirt from an \texttt{XL} t-shirt? The judges certainly couldn't, but it appears that the contestants can when they try putting their new t-shirts on. 
Since the judges never learn to plan properly, this will surely be a problem next year as well. But right now it is \emph{your} problem.

While each contestant has a preferred size, they can wear t-shirts in an interval of sizes. More specifically,
the $i$:th contestant (starting from 0) can wear a t-shirt in any size at least $L[i]$ but at most $H[i]$ (both limits inclusive). 
Here, each size has been assigned an integer so that a higher integer means a larger size.
Your task is to assign t-shirts to the contestants, so that as many contestants as possible will get a t-shirt
that he or she can wear. The judges have ordered exactly $N$ t-shirts, and the size of the $i$:th t-shirt is $T[i]$.

\section*{Example}
There are $N = 3$ contestants. The first contestant can wear t-shirts between sizes $3$ and $7$,
the second contestant can wear t-shirts in sizes $3$ to $5$, and the last contestant can only wear a size $6$ t-shirt.

The three t-shirts have sizes $4, 6, 8$. Since nobody can wear the last t-shirt, at most two t-shirts
can be assigned to the contestants. This can also be achieved, for example by giving the first t-shirt (size $4$) to
the first contestant, and the second t-shirt (size $6$) to the last contestant. Thus the answer is 2.

\section*{Task}
Your task is to compute the maximum number of contestants who can get a t-shirt that he or she can wear.
You should implement the function \texttt{tshirt(N, L, H, T)}.
\begin{itemize}
  \item \texttt{tshirt(N, L, H, T)} - this function will be called exactly once by the judge.
  \begin{itemize}
    \item \texttt{N}: an integer - the number of contestants (as well as t-shirts).
    \item \texttt{L}: an array with $N$ integers. \texttt{L[i]} ($0 \le i < N$) is the smallest size the $i$:th contestant can wear.
    \item \texttt{H}: an array with $N$ integers. \texttt{H[i]} ($0 \le i < N$) is the largest size the $i$:th contestant can wear.
    \item \texttt{T}: an array with $N$ integers. \texttt{T[i]} ($0 \le i < N$) is the size of the $i$:th t-shirt.
    \item $0 \le L[i] \le H[i] \le \max(T)$
    \item The function should return an integer, the maximum number of contestants who can get a t-shirt assigned to them.
  \end{itemize}
\end{itemize}

A code skeleton containing the function to be implemented, together with a sample grader, can be found at
\url{http://progolymp.se/uploads/kattis-attachments/tshirts.zip}.

\section*{Subtasks}
The problem consists of a number of subtasks. Each subtask gives some amount of points, and to pass
the subtask you must pass all the test cases in the subtask.

\begin{tabular}{|l|l|l|}
  \hline
  \textbf{Subtask} & \textbf{Points} & \textbf{Limits} \\ \hline
  1 & 7 & $1 \le N \le 10$, $0 \le T[i] \le 1\,000$ \\ \hline
  2 & 24 & $1 \le N \le 1\,000$, $0 \le T[i] \le 1\,000$ \\ \hline
  3 & 13 & $1 \le N \le 100\,000$, $0 \le T[i] \le 1$ \\ \hline
  4 & 37 & $1 \le N \le 100\,000$, $0 \le T[i] \le 100,000$ \\ \hline
  5 & 19 & $1 \le N \le 100\,000$, $0 \le T[i] \le 10^9$ \\ \hline
\end{tabular}

\section*{Input format}
The sample judge reads input in the following format:

\begin{itemize}
  \item line $1$: \texttt{N}
  \item line $2$: \texttt{L[0] L[1] .. L[N - 1]}
  \item line $3$: \texttt{H[0] H[1] .. H[N - 1]}
  \item line $4$: \texttt{T[0] T[1] .. T[N - 1]}
\end{itemize}

\section*{Output format}
The judge writes a single line containing the return value of \texttt{tshirt(N, L, H, T)}.
