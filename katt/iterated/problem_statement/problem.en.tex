\newcommand\version{v1}
\problemname{Iterated Quiz}
Ann Britt-Caroline is playing a game against Daniel. Daniel have asked Ann $N$ yes-or-no questions, which she now must answer.
However, the questions are seemingly impossible, e.g.
\begin{itemize}
  \item What is the airspeed velocity of an unladen swallow?
  \item Is the statement ``This statement is false'' true?
  \item Is $P = NP$?
  \item ...
\end{itemize}
Very hard questions indeed. To make the game a bit more fun (and to put the \emph{Iterated} in \emph{Iterated Quiz}), Daniel allows Ann to answer all the questions
many times. For a number of rounds, she will answer all of the $N$ questions simultaneously. Daniel will then tell her how many questions she got right.

The goal of the game is to use as few rounds as possible to answer all the questions correctly.
Help her play Iterated Quiz against Daniel!

\section*{Example}
Let there be $N = 3$ questions. In the first round, Ann answers $(yes, yes, no)$. Daniel says that she got $2$ questions correct. She then answers $(yes, no, yes)$. This time, she still got two questions correct. Finally, she
answers $(yes, no, no)$, which happened to be the correct answer. Thus she needed 3 rounds to complete the game.

Note that in the first round, the first and third answers were correct. In the second roun, the first and second answers were correct.

\section*{Task}
Your task is to play Iterated Quiz against Daniel. You will get the number of questions in the game, and should implement the function \texttt{init(N)}.
\begin{itemize}
  \item \texttt{init(N)} - this function will be called exactly once by the judge, at the start of the game.
    \begin{itemize}
      \item \texttt{N}: the number of questions that Daniel asks.
      \item The function has no return value.
    \end{itemize}
\end{itemize}

Additionally, you should call the function \texttt{guess(A)} to give your guesses.
\begin{itemize}
  \item \texttt{guess(A)} - this function should be called by your program to give a guess.
    \begin{itemize}
      \item \texttt{A}: an array with $N$ elements. $A[i]$ ($0 \le i < N$) should contain 0 or 1 if your answer to the $i$:th question is $0$ or $1$ respectively.
      \item The function returns an integer; the number of questions you got right with your guess. When you have answered all questions correctly, your program should terminate.
    \end{itemize}
\end{itemize}


\section*{Subtasks}
The problem consists of a number of subtasks. Each subtask gives some amount of points, and to pass
the subtask you must pass the single test case in the subtask.

Let $M$ be the number of rounds your program needs to answer all the questions correctly.
\begin{tabular}{|l|l|l|}
  \hline
  \textbf{Subtask} & \textbf{Points} & \textbf{Limits} \\ \hline
  1 & 9 & $1 \le N \le 20$. $M$ must be at most $2^N$.  \\ \hline
  2 & 11 & $1 \le N \le 1\,000$, $M$ must be at most $2N$. \\ \hline
  3 & 15 & $1 \le N \le 1\,000$, $M$ must be at most $N + 2$. \\ \hline
  4 & 65 & $500 \le N \le 1\,000$. \\ \hline
\end{tabular}

For subtask 4, the scoring works as follows. Assume that your program needs $M$ rounds,
and that the best program needs $M'$ rounds.

Then you will score
\[ 65 \cdot \max \{ 0, \frac{N - M}{N - M'} \} \]
points on the subtask.

\section*{Input format}
The sample judge reads input in the following format:

\begin{itemize}
  \item line $1$: \texttt{N}
  \item line $2$: \texttt{G[0] G[1] .. G[N - 1]} - \texttt{G[i]} contains the answer to question $i$.
\end{itemize}

\section*{Output format}
The judge will write a single line with the number of guesses your program needs.
