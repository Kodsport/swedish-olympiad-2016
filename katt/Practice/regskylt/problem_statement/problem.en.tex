\newcommand\version{v1}
\problemname{Number Plates}
Simon is in charge of a top secret spy network.
Often when you're a spy, you are yourself spied upon.
Simon's enemies usually spy on him from their cars, but unfortunately he doesn't know what cars they are in.

For every car he sees, he records its number plate on his cell phone.
If he's been seeing a certain number plate very often recently, there's a good chance the car is being used to spy on Simon.
Thus, he needs you to implement a program that tells him how many times he's seen a certain number plate.

However, there is an additional complication.
Simon can sometimes only see a partial plate, for example when the car he is observing is moving very fast.
In this case, he wants to know the total number of times he's seen all the plates that match the partial plate.

A number plate is a string of exactly 6 digits $0-9$. We say that a new observed plate $X$
matches a previously observed plate $Y$, if for every possible plate that $Y$ could have been, it is also
possible that $X$ could be that plate. In particular, if Simon observes the new plate \texttt{00x000}, it
will not match a previously observed plate \texttt{00xx00}, since the latter could have been \texttt{000100}
while the new plate could not. However, if \texttt{00xx00} was the new plate and \texttt{00x000} was the old
they would have matched.

\section*{Example}
Say that the first plate Simon sees is \texttt{000000}. The answer is $0$, since he's never seen a plate before.
The next plate is yet again \texttt{000000}, which he has seen once before. Thus the answer is $1$.
The third plate he sees is \texttt{001000}, which he has never seen, making the answer $0$.
The fourth plate he sees is only partial: \texttt{00x000}, where \texttt{x} denotes a digit he didn't see.
This can match both the plates he's previously seen, so in this case the answer is $3$.
The fifth plate he sees is also partial: \texttt{00xx00}.
Note that this partial plate matches partial plate Simon just saw - the partial plate should then be included as well (making the answer $4$).
Finally, Simon sees \texttt{00x000}, which matches $4$ previous plates (see the explanation above).

\section*{Task}
Given all the plates (either partial or complete) that Simon sees, determine how many of the
plates he've seen before could match the plate he just saw.

You should implement this as the function \texttt{plate(P)}.
\begin{itemize}
  \item \texttt{plate(P)} - this function will be called for every plate that Simon sees.
  \begin{itemize}
    \item \texttt{P}: an array with exactly 6 elements, where \texttt{P[i]} denotes the $i$:th digit in the plate Simon saw.
      If he didn't see the $i$:th digit, it will instead be $-1$.
    \item The function should return the number of previous plates that this one could match.
  \end{itemize}
\end{itemize}

A code skeleton containing the function to be implemented, together with a sample grader, can be found at
\url{http://progolymp.se/uploads/kattis-attachments/regskylt.zip}.

\section*{Subtasks}
The problem consists of a number of subtasks. Each subtask gives some amount of points, and to pass
the subtask you must pass all the test cases in the subtask.

Let $P$ be the number of calls to \texttt{plate}.

\begin{tabular}{|l|l|l|}
  \hline
  \textbf{Subtask} & \textbf{Points} & \textbf{Limits} \\ \hline
  1 & 13 & $P \le 100\,000$, $P[i] \not= -1$ \\ \hline
  2 & 31 & $P \le 100\,000$, $P[i]$ is either $0$ or $-1$ \\ \hline
  3 & 20 & $P \le 1\,000$ \\ \hline
  4 & 36 & $P \le 100\,000$ \\ \hline
\end{tabular}

\section*{Input format}
The sample judge reads input in the following format:

\begin{itemize}
  \item line $1$: \texttt{N} - the number of plates.
  \item lines $2$ to $2 + N - 1$: \texttt{P[0] P[1] .. P[5]}
\end{itemize}

\section*{Output format}
For every plate, the judge writes a line with the return value of \texttt{plate(P)}.
