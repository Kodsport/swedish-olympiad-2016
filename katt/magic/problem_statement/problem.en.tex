\newcommand\version{v1}
\problemname{Magic Show}
Mårten the Magician is currently performing in a magnificent magic competition.
The show consists of $N$ rounds.
In each round, Mårten uses his magic to perform one of two tricks: either he makes
some rabbits appear, or he sabotages his opponents' tricks by making some of
their rabbits disappear. He can also choose to do neither.

For each rabbit Mårten makes appear or disappear, he must use 1 magick.
In the beginning of the show, Mårten has $K$ magicks. When he has run out
of magicks, he can no longer perform a trick.

The scoring of the competition is easy. In each round, let $x$ be the
number of rabbits Mårten made either appear or disappear, where $x$ is negative if he made
the rabbits disappear and positive if he made them appear. If he didn't perform a trick,
$x$ will be zero.
In round $i$, if $x$ is in the interval $[L[i], R[i]]$ where $L[i]$ and $R[i]$ are integers specific
to the round, he gets $|x - \frac{L[i] + R[i]}{2}|$ points. If $x$ is not in this interval,
Mårten gets no score. Note that Mårten cannot perform his magic on fractional rabbits,
thus $x$ must always be an integer.

Mårten's total score in the competition is the sum of scores among all the rounds.
What is the maximum score Mårten can get if he performs optimally?

\section*{Example}
Consider a competition with $N = 4$ rounds, where Mårten has $K = 5$ magicks at the start of the show.
The round constants are $3, 5$ for the first round, $-2, 2$ for the second round, $-2, 0$ for the third
round, and $2, 6$ for the last round.

The optimal play from Mårten is then to do nothing in the first round ($0$ points), remove two rabbits in the second ($|-2 - 0| = 2$ points, 
do nothing in the third round ($|0 - (-1)| = 1$ point), and create 2 rabbits in the last round ($|2 - 4| = 2$ points).

This totals $0 + 2 + 1 + 2 = 5$ points, which is maximal. Note that there are several optimal strategies in the example.
In total, he used $0 + 2 + 0 + 2 = 4$ magicks, leaving an unused magick after the last round.

\section*{Task}
Given all the rounds of the competition, you are to determine the maximum score Mårten can get,
and construct instructions for him.

You should implement the function \texttt{magic\_score(N, K, L, R)}.
\begin{itemize}
  \item \texttt{magic\_score(N, K, L, R)} - this function will be called exactly once by the judge.
  \begin{itemize}
    \item \texttt{N}: the number of rounds in the competition.
    \item \texttt{K}: the amount of magicks Mårten has in the beginning of the show.
    \item \texttt{L}: an array with $N$ elements, containing the values \texttt{L[i]} as descriped in the task.
    \item \texttt{R}: an array with $N$ elements, containing the values \texttt{R[i]} as descriped in the task.
    \item $-K \le L[i] \le R[i] \le K$
    \item $L[i] + R[i]$ is even
    \item The function should return the maximum score Mårten can obtain.
  \end{itemize}

\end{itemize}

Additionally, you should call the function \texttt{trick(X)} to tell Mårten what he should do in each round.
\begin{itemize}
  \item \texttt{trick(X)} - this function should be called by your program once for every round. The $i$:th call should give the instruction
    to Mårten in the $i$:th round if Mårten wants to play optimally.
  \begin{itemize}
    \item \texttt{X}: this parameter should give the move $x$ Mårten should make in the round as described in the task. If Mårten should add rabbits,
      this value should be the number of rabbits Mårten should add. If he should remove rabbits, it should be the negative
      value of the number of rabbits he removes. If he should not do any trick in the round, the value should be $0$. 
    \item The function has no return value.
  \end{itemize}
\end{itemize}


\section*{Subtasks}
The problem consists of a number of subtasks. Each subtask gives some amount of points, and to pass
the subtask you must pass all the test cases in the subtask.

For each subtask, if the return value of \texttt{magic\_score(N, K, L, R)} is correct, but your calls to \texttt{trick(X)}
is not a valid sequence giving the maximum score, you will get $75\%$ of the score of the subtask. This will show up as
\texttt{Partially correct} on these test cases. \textbf{Note that you always must make exactly $N$ calls to \texttt{trick(X)},
even if the sequence is incorrect}.

TODO limits on K

\begin{tabular}{|l|l|l|}
  \hline
  \textbf{Subtask} & \textbf{Points} & \textbf{Limits} \\ \hline
  1 & 20 & $N \le 100$ \\ \hline
  2 & 20 & $N \le 1\,000$ $L[i] \ge 0$\\ \hline
  3 & 20 & $N \le 1\,000$, $R[i] = 2 + L[i]$ \\ \hline
  4 & 40 & $N \le 1\,000$ \\ \hline
\end{tabular}

\section*{Input format}
The sample judge reads input in the following format:

\begin{itemize}
  \item line $1$: \texttt{N K}
  \item line $2$: \texttt{L[0] L[1] .. L[N - 1]}
  \item line $3$: \texttt{R[0] R[1] .. R[N - 1]}
\end{itemize}

\section*{Output format}
The sample judge writes output in the following format:

\begin{itemize}
  \item line $1$: the return value of \texttt{magic\_score(N, K, L, R)} on a line
  \item line $2$: $N$ integers, the values given from the calls to \texttt{trick(X)} in order.
\end{itemize}
