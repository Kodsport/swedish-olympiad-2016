\newcommand\version{v1}
\problemname{Fibonacci Strings}
Aron likes Fibonacci numbers. He likes them so much he actually got bored of the numbers themselves and decided to invent
other combinatorial objects based on them instead.

His first invention is the \emph{Fibonacci string}. We call a string consisting of only letters $a$ and $b$,
with exactly $n$ letters $a$ without any two consecutive letters $a$ a \emph{Fibonacci string of the $n$:th order}.

Given a string $X$ of $a$:s and $b$:s, compute the sum of orders of all Fibonacci strings that are substrings of $X$.
Note that if a string $X$ appears multiple times, its order should be counted once for every occurance.

\section*{Example}
Assume that Aron has the string $abaaba$. This contains the Fibonacci string $a$ four times, $ab$ two times, $ba$ two times and $aba$ two times.
The sum of their orders is $4 \cdot 1 + 2 \cdot 1 + 2 \cdot 1 + 2 \cdot 2 = 4 + 2 + 2 + 4 = 12$.

\section*{Task}
Write a program that given $X$ computes the sum of orders of all Fibonacci substrings of $X$.
You should implement the function \texttt{fibonacci(N, X)}.

\begin{itemize}
  \item \texttt{fibonacci(N, X)} - this function will be called exactly once by the judge.
  \begin{itemize}
    \item \texttt{N}: an integer - the number of charaters in the string $X$.
    \item \texttt{X}: a string consisting of $X$ characters, consisting only lowercase letters $a$ and $b$.
    \item The function should return an integer, the sum of orders of all Fibonacci substrings of $X$.
  \end{itemize}
\end{itemize}

\section*{Subtasks}
The problem consists of a number of subtasks. Each subtask gives some amount of points, and to pass
the subtask you must pass all the test cases in the subtask.

\begin{tabular}{|l|l|l|}
  \hline
  \textbf{Subtask} & \textbf{Points} & \textbf{Limits} \\ \hline
  1 & 9 & $1 \le N \le 100$  \\ \hline
  2 & 11 & $1 \le N \le 1\,000$  \\ \hline
  3 & 14 & $1 \le N \le 100\,000$. Each letter of $X$ was chosen randomly with uniform probability.  \\ \hline
  4 & 25 & $1 \le N \le 100\,000$. $X$ is a Fibonacci string.  \\ \hline
  5 & 45 & $1 \le N \le 100\,000$  \\ \hline

\end{tabular}

\section*{Input format}
The sample judge reads input in the following format:

\begin{itemize}
  \item line $1$: \texttt{N}
  \item line $2$: \texttt{X}
\end{itemize}

\section*{Output format}
The judge writes a single line containing the return value of \texttt{fibonacci(N, X)}.
