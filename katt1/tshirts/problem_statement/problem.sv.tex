\newcommand\version{v1}
\problemname{T-shirts}
Under PO-finalen får varje av de $N$ deltagarna en t-shirt. Domarna har dock en tendens att vara lite
stressade med sistaminutenändringar till problemen (men bara ändringar -- domarna väntar aldrig
till dagen innan tävlingen med att göra ett nytt problem).

Detta betyder att när domarna beställer t-shirts kanske de inte tittar \emph{jättenoga} på vilka
t-shirtstorlekar deltagarna har. När allt kommer omkring, vem kan faktiskt
skilja på en \texttt{XS}-t-shirt och en \texttt{XL}-t-shirt. Domarna kan det uppenbarligen inte, men
det verkar som att de tävlanden kan när de sätter på sig sina nya t-shirts.
Eftersom domarna aldrig lär sig att planera ordentligt kommer detta säkerligen vara ett problem nästa år också.
Men just nu är det \emph{ditt} problem.

Varje tävlanden har en föredragen storlek, men de kan faktiskt ha t-shirts i ett visst interval av storlekar.
Mer specifikt kan den $i$:te tävlanden (noll-indexerat) bära en t-shirt som har minst storlek $L[i]$ och högst
storlek $H[i]$ (båda gränser inklusive).
Här har varje storlek blivit tilldelat ett heltal, så att ett större tal innebär en större storlek.
Din uppgift är att tilldela t-shirts till deltagarna, så att så många deltagare som möjligt får en
t-shirt som hen kan ha på sig. Domarna har beställt exakt $N$ t-shirts, och storleken på den $i$:te t-shirten
är $T[i]$.

\section*{Exempel}
Vi har $N = 3$ deltagare. Den första deltagaren kan ha t-shirts i storlekar mellan $3$ och $7$, 
den andra i storlekar mellan $3$ och $5$, och den sista tävlanden kan bara ha en t-shirt av storlek $6$.

De tre t-shirtarna har storlekarna $4, 6, 8$. Eftersom ingen kan bära den sista t-shirten kan högst
två tävlanden få t-shirts. Vi kan också tilldela så många t-shirts, till exempel genom att ge den
första t-shirten (storlek $4$) till den första tävlanden, och den andra t-shirten (storlek $6$) till den
sista tävlanden. Svaret är således 2.

\section*{Uppgift}
Din uppgift är att beräkna det maximala antalet deltagare som kan få en t-shirt som hen kan ha på sig.
Du ska implementera funktionen \texttt{tshirt(N, L, H, T)}.

\begin{itemize} 
  \item \texttt{tshirt(N, L, H, T)} - denna funktion kommer anropas exakt en gång av domaren.
  \begin{itemize}
    \item \texttt{N}: ett heltal - antalet tävlanden (och t-shirts).
    \item \texttt{L}: en array med $N$ heltal. \texttt{L[i]} ($0 \le i < N$) är den minsta storleken den $i$:te tävlanden kan ha på sig.
    \item \texttt{H}: en array med $N$ heltal. \texttt{H[i]} ($0 \le i < N$) är den största storleken den $i$:te tävlanden kan ha på sig.
    \item \texttt{T}: en array med $N$ heltal. \texttt{H[i]} ($0 \le i < N$) är storleken på den $i$:te t-shirten.
    \item $0 \le L[i] \le H[i] \le \max(T)$, där $\max(T)$ betecknar det största värdet i arrayen $T$.
    \item Funktionen ska returnera ett heltal - det största antalet tävlanden som kan få en t-shirt tilldelad till sig.
  \end{itemize}
\end{itemize}

Ett kodskelett som innehåller funktionen du ska implementera, tillsammans med en exempeldomare, finns tillgängligt på
\url{http://progolymp.se/uploads/kattis-attachments/tshirts.zip}.


\section*{Delpoäng}
Uppgiften består av ett antal grupper. Varje grupp ger ett visst antal poäng, och för att klara
gruppen måste du klara samtliga testfall i gruppen.

Låt $J$ vara antalet anrop till \texttt{jump} och $S$ antalet anrop till \texttt{score}.

\begin{tabular}{|l|l|l|}
  \hline
  \textbf{Grupp} & \textbf{Poäng} & \textbf{Gränser} \\ \hline
  1 & 7 & $1 \le N \le 10$, $0 \le T[i] \le 1\,000$ \\ \hline
  2 & 24 & $1 \le N \le 1\,000$, $0 \le T[i] \le 1\,000$ \\ \hline
  3 & 13 & $1 \le N \le 100\,000$, $0 \le T[i] \le 1$ \\ \hline
  4 & 37 & $1 \le N \le 100\,000$, $0 \le T[i] \le 100,000$ \\ \hline
  5 & 19 & $1 \le N \le 100\,000$, $0 \le T[i] \le 10^9$ \\ \hline
\end{tabular}

\section*{Indataformat}
Exempeldomaren läser indata i följande format:

\begin{itemize}
  \item rad 1: \texttt{N}
  \item rad 2: \texttt{L[0] L[1] ... L[N - 1]}
  \item rad 3: \texttt{H[0] H[1] ... H[N - 1]}
  \item rad 4: \texttt{T[0] T[1] ... T[N - 1]}
\end{itemize}

\section*{Utdataformat}
Exempeldomaren skriver ut en enda rad med returvärdet av \texttt{tshirt(N, L, H, T)}.
