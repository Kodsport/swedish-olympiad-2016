\newcommand\version{v2}
\problemname{Magishow}
Magikern Mårten håller just nu på att uppträda i en magnifik magitävling.
Tävlingen består av $N$ rundor.
I varje runda använder Mårten sin magi för att göra ett av två trick: antingen trollar
han fram $x$ kaniner, eller så saboterar han ett av sina motståndares trick genom att
trolla bort $y$ av deras kaniner. Han kan också välja att inte göra något.

För varje kanin som Mårten trollar fram eller trolla bort måste han använda 1 magi.
I början av showen har han $K$ magi. När han har slut på magi så kan han inte längre
utföra trick.

Poängsättningen i tävlingen är enkel. I runda $i$, låt
\[ S_i = \begin{cases}
  x & \text{ om Mårten trollade fram $x$ kaniner } \\
  -y & \text{ om Mårten trollade bort $y$ kaniner } \\
  0 & \text{ om Mårten inte utförde ett trick } \\
\end{cases}
\]

I runda $i$, om $S_i$ är i intervallet $[L[i], R[i]]$ där $L[i]$ och $R[i]$ är heltal specifika
till runda $i$, så får han $|S_i - \frac{L[i] + R[i]}{2}|$. Om $S_i$ är utanför intervallet
får han istället $0$ poäng. Notera att Mårten endast kan utföra sin magi på hela kaniner,
så $S_i$ kommer alltid att vara ett heltal.

Mårtens totala poäng i tävlingen är summan av hans poäng för varje runda.
Vad är den maximala poängen Mårten kan uppnå?

\section*{Exempel}
Betrakta en tävling med $N = 4$ rundor, där Mårten har $K = 5$ magi i början av tävlingen.
Intervallen för rundorna är $[3, 5]$ för första rundan, $[-2, 2]$ för andra rundan, $[-2, 0]$ för tredje rundan,
och $[2, 6]$ för sista rundan.

Mårtens optimala trick är då att i första rundan inte göra något ($0$ poäng), trolla bort två kaniner i andra rundan ($|-2 - 0 | = 2$ poäng),
inte göra något i tredje rundan ($|0 - (-1)| = 1$ poäng) och trolla fram 2 kaniner i sista rundan ($|2 - 4| = 2$ poäng).

Totalt blir detta $0 + 2 + 1 + 2 = 5$ poäng, vilket är den bästa möjliga poängen. Notera att det finns flera optimala strategier i detta exempel.
Totalt använde han $0 + 2 + 0 + 2 = 4$ magi, så Mårten har en magi över efter sista rundan.

\section*{Uppgift}
Givet alla rundorna i tävlingen, bestäm den maximala poängen Mårten kan få, och
ge honom instruktioner för hur han uppnår den.

Du ska implementera funktionen \texttt{magic\_score(N, K, L, R)}.
\begin{itemize}
  \item \texttt{magic\_score(N, K, L, R)} - denna funktion anropas exakt en gång av domaren.
  \begin{itemize}
    \item \texttt{N}: antalet rundor i tävlingen.
    \item \texttt{K}: mängden magi Mårten har i början av tävlingen.
    \item \texttt{L}: en vektor med $N$ element, som innehåller värdena \texttt{L[i]} som beskrivs i uppgiften.
    \item \texttt{R}: en vektor med $N$ element, som innehåller värdena \texttt{R[i]} som beskrivs i uppgiften.
    \item $-1\,000\,000 \le L[i] \le R[i] \le 1\,000\,000$
    \item $L[i] + R[i]$ är jämnt
    \item Funktionen ska returnera den maximala poäng som Mårten kan få.
  \end{itemize}

\end{itemize}

Du ska dessutom anropa funktionen \texttt{trick(X)} för att berätta för Mårten vad han ska göra i varje runda.
\begin{itemize}
  \item \texttt{trick(X)} - denna funktion ska anropas en gång per runda av ditt program.
    Det $i$:te anropet ska ge instruktionen till Mårten för den $i$:te rundan, om han ska spela optimalt.
  \begin{itemize}
    \item \texttt{X}: denna parameter ska ge värdet $S_i$ på varje av Mårtens rundor.
      Om Mårten ska trolla fram kaniner,
      så ska detta värde vara antalet kaniner Mårten ska lägga till. Om han ska trolla bort kaniner ska detta vara minus antalet kaniner Mårten trollade bort. 
      Om han inte ska utföra något trick ska värdet vara $0$.
    \item Funktionen har inget returvärde.
  \end{itemize}
\end{itemize}

Ett kodskelett som innehåller funktionen du ska implementera, tillsammans med en exempeldomare, finns tillgängligt på
\url{http://progolymp.se/uploads/kattis-attachments/magic.zip}.

\section*{Delpoäng}
Uppgiften består av ett antal grupper. Varje grupp ger ett visst antal poäng, och för att klara
gruppen måste du klara samtliga testfall i gruppen.

För varje delgrupp, om du svarar rätt returvärde på \texttt{magic\_score(N, K, L, R)} är rätt men dina anrop till \texttt{trick(X)}
inte är en giltig sekvens som ger maximal poäng, så får du $75\%$ av gruppens poäng.
Detta kommer synas som \texttt{Partially correct} on de testfall där så är fallet. \textbf{Notera att du alltid måste göra $N$ anrop till \texttt{trick(X)},
även om sekvensen inte är korrekt}.

\begin{tabular}{|l|l|l|}
  \hline
  \textbf{Grupp} & \textbf{Poäng} & \textbf{Gränser} \\ \hline
  1 & 20 & $N \le 100$, $K \le 100$ \\ \hline
  2 & 20 & $N \le 1\,000$, $K \le 1,000$ $L[i] \ge 0$\\ \hline
  3 & 20 & $N \le 1\,000$, $K \le 1,000$, $R[i] = 2 + L[i]$ \\ \hline
  4 & 40 & $N \le 1\,000$, $K \le 1,000$ \\ \hline
\end{tabular}

\section*{Indataformat}
Exempeldomaren läser indata i följande format:

\begin{itemize}
  \item rad $1$: \texttt{N K}
  \item rad $2$: \texttt{L[0] L[1] .. L[N - 1]}
  \item rad $3$: \texttt{R[0] R[1] .. R[N - 1]}
\end{itemize}

\section*{Utdataformat}
Exempeldomaren skriver utdata på följande format:

\begin{itemize}
  \item rad $1$: returvärdet av \texttt{magic\_score(N, K, L, R)}
  \item rad $2$: $N$ heltal, värdena som ges av anropen till \texttt{trick(X)} i ordning.
\end{itemize}
