\problemname{Siffersumma}

Universitetslektor Bjarki undervisar i många olika kurser varje vecka, men är
inte särskilt organiserad av sig. Desto mer förvirrad blir han när kurserna
från vecka till vecka har olika antal schemalagda tillfällen.

Om en kurs har $A$ schemalagda tillfällen en vecka, och den nästa vecka har $B$
tillfällen, kommer Bjarki ändå bara att hålla exakt $A$ föreläsningar. Därför
kan det både hända att Bjarki håller lektion inför tomma klassrum samt kan det hända
att han håller färre föreläsningar än det var tänkt. I slutet av veckan
får han dock ett argt brev av skolchefen med vilka tider han skulle hållit
föreläsningar och kommer istället att använda dessa tider veckan därpå.

Givet hur många föreläsningar Bjarki håller under $N$ veckor, bestäm antalet
föreläsningar Bjarki kommer hålla inför tomma klassrum samt antalet
föreläsningar Bjarki inte dyker upp på.

\section*{Exempelkörning}

Se det första indataexemplet. Första veckan har Bjarki alltid koll på vilka
föreläsningar han ska hålla. Veckan därpå kommer han tro att han enbart ska
hålla en föreläsning, och missar därför 2. Därefter kommer han hålla en
föreläsning inför ett tomt klassrum, och sista veckan kommer han missa två
föreläsningar. Totalt har han hållt 1 tom föreläsning och missat 4
föreläsningar.

\section*{Indata}

Först kommer talet $N$ på en egen rad. Därefter kommer $N$ heltal, antalet
lektioner Bjarki skulle ha hållit, vecka för vecka.

Det kan aldrig vara mer än 10 föresläsningar under en vecka och det kan inte
finnas mer än 9 veckor i indatat (en tentatermin).

\section*{Utdata}
Skriv ut antalet tomma föresläsningar Bjarki har hållt, ett mellanslagtecken,
därefter antalet föreläsningar Bjarki har missat.
