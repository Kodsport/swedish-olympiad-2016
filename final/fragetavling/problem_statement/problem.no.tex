\problemname{Quiz}
In the quiz ProgrammeringsQuiz there are $N$ questions in total, distributed over $M$ different categories (for example, algorithm theory, compiler construction or Sven knowledge).

The questions are worth different amounts of points. Additionally, you will get a bonus $B$ if you answer all questions in a certain category.
Simone has participated in Programmeringsolympiaden since 8th grade, so she is able to answer all the questions.

Unfortunately, there is a time limit to the quiz. Despite never giving the wrong answer, Simone will only have time to answer $K$ questions. What is the maximum number of points she can achieve?
\section*{Input}
The first line consists of four integers $1 \le N \le 1000$, $1 \le M \le N$, $1 \le K \le N$, $1 \le B \le 100\,000$.
The following $N$ lines consist of two integers each: the points given for answering the question (an integer between $1$ and $1\,000$) and which category it belongs to (between $1$ and $M$).
Each category will have at least one question.
\section*{Output}
Print one line containing the maximal possible number of points.
\section*{Grading}
Your solution will be tested on several groups of test cases. To get points for a group you need to pass all the tests of that group.

\begin{tabular}{| l | l | l |}
  \hline
  Group & Points & Constraints\\ \hline
  1     & 16         & $N \le 100, M = 1$ \\ \hline
  2     & 29         & $M \le 5$ \\ \hline
  3     & 20         & all question give the same number of points \\ \hline
  4     & 35         & \\ \hline
\end{tabular}
\section*{Explanation}
In the first sample Simone answers both questions from category 1 ($300 + 400 = 700$ points) and the only question in category 2 ($200$ points). Since these were the only questions in these two categories we get two bonuses, which gives a total of $200 + 700 + 2 \cdot 1000 = 2900$ points.
