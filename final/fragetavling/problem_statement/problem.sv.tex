\problemname{Frågetävling}
I frågetävlingen ProgrammeringsQuiz finns det totalt $N$ frågor, fördelade över $M$ olika kategorier (t.ex. algoritmteori,
kompilatorkonstruktion eller Sven-kunskap).

Frågorna är värda olika mycket poäng. Dessutom får du en bonus $B$ när du löser samtliga problem i en viss kategori.
Simone har varit med i Programmeringsolympiaden sen åttonde klass, så hon kommer kunna svara rätt på alla frågor.

Tyvärr går tävlingen på tid. Trots att hon aldrig svarar fel kommer hon bara hinna svara på $K$ av frågorna.
Vilken poäng kan Simone maximalt uppnå?

\section*{Input}
Den första raden innehåller de fyra heltalen $1 \le N \le 1000$, $1 \le M \le N$, $1 \le K \le N$, $1 \le B \le 100\,000$.
De följande $N$ raderna innehåller två tal var: poängen för frågan (ett heltal mellan 1 och $1\,000$) och vilken kategori den ingår i (mellan 1 och $M$).
Det är garanterat att alla kategorier kommer att innehålla minst en fråga.

\section*{Output}
Du ska skriva ut en enda rad med den maximala poängen du kan uppnå.

\section*{Poängsättning}
Din lösning kommer att testas på en mängd testfallsgrupper. För att få poäng för en grupp så måste du klara alla testfall i gruppen.

\begin{tabular}{| l | l | l |}
  \hline
  Grupp & Poängvärde & Begränsningar\\ \hline
  1     & 16         & $N \le 100, M = 1$ \\ \hline
  2     & 29         & $M \le 5$ \\ \hline
  3     & 20         & alla frågor ger lika mycket poäng \\ \hline
  4     & 35         & \\ \hline
\end{tabular}
