\problemname{Solitärschack}

Mårten är bra på schack. Lite för bra, enligt vissa. Till exempel Johan. Mårten utmanar ofta Johan på schack,
vilket leder till att Johan förlorar på schack. Istället för att erkänna sig besegrad gör Johan vad han
brukar göra när han förlorar på något; han ändrar reglerna.

Denna gång har han uppfunnit ett nytt slags enspelarschack som han kallar för \emph{Solitärschack}. Spelet
körs på ett $6 \times 6$ bräde med pjäser. Pjäserna finns i tre olika färger - brons, silver, guld - och
åtta olika typer.

I början är alla pjäser på brädet bronsfärgade. Spelet börjar med att Johan väljer en av pjäserna på brädet.
Denna pjäs ersätts sedan med en silverpjäs, vars typ är slumpmässig. Sedan, beroende på vilken pjäs Johan
valde, får han välja en annan pjäs på brädet att ta bort. Beroende på vilken denna var får han ta bort ytterligare
en annan pjäs, och upprepar denna procedur tills inga fler drag är möjliga.

När en bronspjäs tas bort ersätts den alltid med en slumpmässig silverpjäs. När en silverpjäs tas bort
ersätts den alltid med en slumpmässig guldpjäs. När en guldpjäs tas bort lämnas rutan istället tom.

Låt $(r, c)$ vara rad och kolumn för den pjäs som blev borttagen, och $(r', c')$ rad och kolumn för den
pjäs du vill ta bort härnäst. De olika pjäsernas drag är då:

\begin{description}
  \item[\texttt{1}] nästa pjäs måste vara exakt 1 steg (horisontellt, vertikalt eller diagonalt) från denna pjäs.
    Mer formellt måste någon av följande gälla:
    $$|r - r'| = 1, |c - c'| = 0$$
    $$|r - r'| = 0, |c - c'| = 1$$
    $$|r - r'| = 1, |c - c'| = 1$$

  \item[\texttt{2}] nästa pjäs måste vara exakt 2 steg (horisontellt, vertikalt eller diagonalt) från denna pjäs.
    Mer formellt måste någon av följande gälla:
    $$|r - r'| = 2, |c - c'| = 0$$
    $$|r - r'| = 0, |c - c'| = 2$$
    $$|r - r'| = 2, |c - c'| = 2$$

  \item[\texttt{3}] nästa pjäs måste vara exakt 3 steg (horisontellt, vertikalt eller diagonalt) från denna pjäs.
    Mer formellt måste någon av följande gälla:
    $$|r - r'| = 3, |c - c'| = 0$$
    $$|r - r'| = 0, |c - c'| = 3$$
    $$|r - r'| = 3, |c - c'| = 3$$

  \item[\texttt{4}] nästa pjäs måste vara exakt 4 steg (horisontellt, vertikalt eller diagonalt) från denna pjäs.
    Mer formellt måste någon av följande gälla:
    $$|r - r'| = 3, |c - c'| = 0$$
    $$|r - r'| = 0, |c - c'| = 3$$
    $$|r - r'| = 3, |c - c'| = 3$$

  \item[\texttt{torn}] nästa pjäs måste ligga antingen rakt
    uppåt, nedåt, till vänster eller till höger om denna pjäs, förflyttad ända bort mot kanten.
    Mer formellt måste någon av följande gälla:
    $$r' \in \{1, 6\}, |c - c'| = 0$$
    $$|r - r'| = 0, c' \in \{1, 6\}$$

  \item[\texttt{lopare}] nästa pjäs måste vara längs med brädets sidor, på samma diagonal som denna pjäs.
    Mer formellt måste någon av följande gälla:
    $$|r - r'| = |c - c'|, r' \in \{1, 6\}$$
    $$|r - r'| = |c - c'|, c' \in \{1, 6\}$$

  \item[\texttt{dam}] nästa pjäs måste vara placerad som om denna pjäs var antingen \texttt{torn} eller \texttt{lopare}.
    Mer formellt måste någon av följande gälla:
    $$r' \in \{1, 6\}, |c - c'| = 0$$
    $$|r - r'| = 0, c' \in \{1, 6\}$$
    $$|r - r'| = |c - c'|, r' \in \{1, 6\}$$
    $$|r - r'| = |c - c'|, c' \in \{1, 6\}$$

  \item[\texttt{springare}] nästa pjäs måste vara belägen antingen på raden ovan/under och 2 kolumner vänster/höger, eller
    på kolumnen vänster/höger och 2 rader ovan/under.
    Mer formellt måste någon av följande gälla:
    $$|r - r'| = 2, |c - c'| = 1$$
    $$|r - r'| = 1, |c - c'| = 2$$
\end{description}

Det är inte tillåtet att ha $(r', c') = (r, c)$.

Poängsättningen är som följer. För varje ruta när spelet är över får man följande poäng per ruta:
\begin{description}
  \item[brons] 0 poäng
  \item[silver] 1 poäng
  \item[guld] 2 poäng
  \item[tom] 3 poäng
\end{description}

Man kan också få bonuspoäng under spelet. Följande bonusar finns:
\begin{itemize}
  \item Om du tar bort $N \ge 2$ pjäser i rad av samma typ får du $2N$ poäng.
  \item Om du tar bort pjäserna $1, 2, 3, 4$ i rad i ordning eller omvänd ordning får du 12 poäng.
  \item Om du tar bort pjäserna $1, 2, 3, 4$ i rad i en annan ordning får du 8 poäng.
  \item Om du tar bort pjäserna $torn, lopare, dam, springare$ i rad i någon ordning får du 8 poäng.
  \item Att ta bort siffrorna $1, 2, 3, 4$ i någon ordning kallas för ett \emph{sifferset}.
    Att ta bort pjäsern $torn, lopare, dam, springare$ i någon ordning kallas för ett \emph{pjässet}.

    Om du tar bort $K \ge 2$ stycken sifferset och pjässet i \emph{alternerande ordning}, d.v.s antingen
    sifferset - pjässet - sifferset - ... eller pjässet - sifferset - pjässet - ..., så får du
    ytterligare $8K$ poäng.
\end{itemize}

Bonusar av en och samma sort kan inte överlappa, och de delas ut girigt.
T.ex. skulle sekvensen av borttagna pjäser \texttt{4 1 2 3 4 dam torn springare lopare} ge $8 + 8 = 16$ bonuspoäng:
8 för sekvensen \texttt{4 1 2 3},
inga för sekvensen \texttt{1 2 3 4} eftersom den överlappar med den tidigare,
och 8 för sekvensen \texttt{dam torn springare lopare}.
Inga ytterligare poäng för alternerande pjässet ges, eftersom vi inte fått poäng för sekvensen \texttt{1 2 3 4}.
Sekvensen \texttt{1 1 2 3 4} ger $4 + 12 = 16$ poäng: \texttt{1 1} och \texttt{1 2 3 4} är olika typer av bonusar, så det är okej att de överlappar.

Det visade sig dock vara svårare att få bra poäng i Solitärschack än att slå Mårten på vanlig schack... Hjälp Johan att få så
bra poäng på Solitärschack som möjligt!

\section*{Input}
Indatat börjar med 6 rader, med 6 blankstegsseparerade strängar vardera. Dessa anger vilka de ursprungliga
bronspjäserna är.

\section*{Interaktion}
Detta är ett interaktivt problem. För att göra ett drag skriver du ut två heltal \texttt{r c}, den rad och kolumn som
innehåller den pjäs du vill ta bort.

Du ska sedan läsa in en rad - namnet på den pjäsen som ersatte den du tog bort. Om du just tog bort en guldpjäs kommer raden att innehålla ordet \texttt{blank}.

När du inte vill göra fler drag ska du skriva ut en rad \texttt{0 0} och avsluta programmet.

Glöm inte att flusha standard out när du printat! I Java gör du detta med \texttt{System.out.flush()},
i C med \texttt{fflush(stdout)}, i C++ med \texttt{flush(cout)}, i C\# med \texttt{Console.Out.Flush()}
och i Python med \texttt{sys.stdout.flush()}.

\section*{Poängsättning}
Antag att på ett testfall värt $P$ poäng så får ditt program $X$ poäng, och det bästa programmet $Y$ poäng.
Då får du $P \times \frac{X}{Y}$ poäng. Innan tävlingsslut kommer alla lösningar rättas med $Y = 200$.
Efter tävlingsslut rättas detta om med de bästa deltagarlösningarna på varje testfall.

Testfallen är indelade i ett antal grupper.

\begin{tabular}{| l | l | l |}
  \hline
  Grupp & Poängvärde & Begränsningar\\ \hline
  1     & 4          & Endast pjäsen springare förekommer. \\ \hline
  2     & 4          & Endast pjäsen 1 förekommer. \\ \hline
  3     & 6          & Endast pjäserna 1 och 2 förekommer. \\ \hline
  4     & 9          & Endast pjäserna torn, lopare, dam förekommer. \\ \hline
  5     & 24         & Endast pjäserna 1, 2, 3, 4 förekommer. \\ \hline
  6     & 53         & Alla pjäser kan förekomma. \\ \hline
\end{tabular}
