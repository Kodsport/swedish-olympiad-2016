\problemname{Godis}

Det är lördag, och Johan ska handla godis. Han har identifierat ett antal olika godispåsar han kan tänka sig att köpa.

Varje påse innehåller ett antal godisbitar av olika sorter. Det finns 10 olika sorters vanligt godis (dessa är numrerade 1 .. 10), och 10 sorter av en ny typ av anti-godis (numrerade -1 .. -10). Det händer sig nu att godisbitar av sort n och av sort -n inte går särskilt väl ihop - de annihileras när de kommer i kontakt med varandra. I övrigt har dock anti-godiset samma konsistens och smak som vanligt godis. Hur många godisbitar kan Johan ha kvar i slutet (efter att godis/anti-godis tagit ut varandra), om han väljer godispåsar på bästa sätt? Ignorera eventuella pengabekymmer - hans föräldrar bjuder.

\section*{Input}

Den första raden i indatan innehåller ett heltal $1 \le N \le 1000$ - antalet godispåsar.

De följande $N$ raderna beskriver varje godispåse.
Först på raden finns ett tal $1 \le k \le 10$, som specificerar antalet olika sorters godis i påsen.
Sedan följer $k$ par av tal $s n$, vilket betyder att det finns $n$ godisar av sort $s$.
Varje sort nämns högst en gång per godispåse, och sorterna $s$ och $-s$ kan av naturliga skäl inte förekomma tillsammans i samma påse.

Alla $n$ kommer att uppfylla $1 \le n \le 1000$.

\section*{Output}

Skriv ut ett tal: den största mängden godis Johan kan ha i slutändan.

% förklaring av sample: att köpa påsar 1 och 2 ger två godisar kvar av sort 1, och fem av sort -2

\section*{Poängsättning}
Din lösning kommer att testas på en mängd testfallsgrupper. För att få poäng för en grupp så måste du klara alla testfall i gruppen.

\begin{tabular}{| l | l | l |}
	\hline
	Grupp & Poängvärde & Begränsningar\\ \hline
  1     & 25         & $N \le 15$ \\ \hline
  2     & 25         & $N \le 1000$. Varje påse innehåller bara en sorts godis. \\ \hline
  3     & 50         & $N \le 1000$ \\ \hline
\end{tabular}
