\problemname{Candy}

It is Saturday and Ann Britt-Caroline is going to buy candy. She has identified several different bags of candy she is considering buying.

Each bag contains a number of pieces of candy of different types. There are 10 types of normal candy (these are numbered $1...10$), and 10 types of anti-candy (numbered $-1...-10$). It just so happens that candy of type $n$ and type $-n$ do not go well together - they are annihilated if they come in contact with each other. Other than that anti-candy tastes just the same as normal candy.

When Ann Britt-Caroline has bought the bags of candy she mixes them in a big bowl such that all pairs of candy/anti-candy is annihilated (she mixes thoroughly). How many pieces of candy can Ann Britt-Caroline have left (after all pairs of candy/ant-candy is annihilated), if she selects her bags of candy optimally? Note that she can only buy one bag of each type. Ignore any money related issues - her parents will pay.
\section*{Input}

The first line in the input consist of an integer $1 \le N \le 1000$ - the number of bags of candy.

The following $n$ lines describe each bag of candy.
Each line starts with an integer $1 \le k \le 10$, specifying the number of different types of candy in the bag.
Then follow $k$ pairs of integers $s$ $n$, which means that there are $n$ pieces of candy type $s$.
Each type of candy is mentioned at most once per bag, and types $s$ and $-s$ can not be in the same bag.

For all $n$, $1 \le n \le 1000$.
\section*{Output}

Print an integer: the largest amount of candy Ann Britt-Caroline can have in the end.
\section*{Grading}
Your solution will be tested on several groups of test cases. To get points for a group you need to pass all the tests of that group.

\begin{tabular}{| l | l | l |}
	\hline
	Group & Points & Constraints\\ \hline
  1     & 25         & $N \le 15$ \\ \hline
  2     & 25         & $N \le 1000$. each bag will contain only one type of candy. \\ \hline
  3     & 50         & $N \le 1000$ \\ \hline
\end{tabular}
