\problemname{Rektangelmagi}

En \emph{magisk sekvens} av längd $n + 1$ är en sekvens av tal $a, a + d, a + 2d, ..., a + nd$ för två rationella tal $a$ och $d$, t.ex. $2, 5.5, 9, 12.5$ eller $5, 5, 5$ eller $2, 1, 0, -1, -2$.

En \emph{magisk rektangel} av storlek $R \times C$ är en rektangel där varje rad och kolumn är en magisk sekvens.

Givet en rektangel av heltal där vissa av talen är bortsuddade, avgör om det går att fylla i dessa bortsuddade tal så att rektangeln är en magisk rektangel.

\section*{Indata}
Den första raden i indata innehåller talen $R$ och $C$, antalet rader och kolumner i den givna rektangeln. Sedan följer $R$ rader med $C$ heltal vardera.

Ett bortsuddat tal representeras som en punkt.

\section*{Utdata}
Om ingen lösning finns, skriv ut \texttt{ej magisk}.

Annars, skriv ut $R$ rader med $C$ kolumner - en magisk rektangel där du tagit indatarektangeln och ersatt de bortsuddade talen.

Rationella tal ska anges på formen \texttt{N/D}, där $N$ och $D$ är högst 100 siffror långa. Observera att det inte ska vara mellanslag mellan talen och divisionstecknet.

Om $D = 1$ kan du skriva \texttt{N}.

\section*{Poängsättning}
Din lösning kommer att testas på en mängd testfallsgrupper.
För att få poäng för en grupp så måste du klara alla testfall i gruppen.

I fall 1-9 gäller $1 \le R, C \le 6$.

\noindent
\begin{tabular}{| l | l | p{12cm} |}
  \hline
  \textbf{Grupp} & \textbf{Poäng} & \textbf{Gränser} \\ \hline
  $1$    & $10$         & Alla tal är redan ifyllda.  \\ \hline
  $2$    & $10$         & Antingen $R$ eller $C$ är 1. \\ \hline
  $3$    & $10$         & $R = C = 2$ \\ \hline
  $4$    & $10$         & Varje testfall har en unik lösning, och rektangeln är konstruerad så att det finns en rad eller kolumn med bara ett bortsuddat tal, och när den fylls i finns det återigen en rad eller kolumn med bara ett tal, osv, ända tills hela rektangeln är ifylld.  \\ \hline
  $5$    & $10$         & Varje testfall har en unik lösning som innehåller enbart heltal. \\ \hline
  $6$    & $10$         & Varje testfall har en unik lösning. \\ \hline
  $7$    & $10$         & Varje testfall har antingen en unik lösning som innehåller enbart heltal, eller så har det inte en lösning. \\ \hline
  $8$    & $10$         & Varje testfall har antingen en unik lösning eller ingen lösning alls. \\ \hline
  $9$    & $10$         & Inga ytterligare begränsningar. \\ \hline
  $10$    & $10$         & Inga ytterligare begränsningar, men $1 \le R, C \le 50$. \\ \hline

\end{tabular}

